%. . . . . . . . . . . . . . . . . . . . . . . . . . . . . . . . . . . . . . . .
% uc-132 Temporizador Recabando lista de proxy svc expuestos en OSB}
\subsection{\ucCXxxiiFullName{}}\label{uc-132}
\textcolor{gray}{V\'ease la realizaci\'on del caso de uso, sec. \ref{sec:ucr-132}}

\subsubsection*{Descripci\'on}
\paragraph{}
Peri\'odicamente el Temporizador solicita al Cat\'alogo
que actualice su registro de servicios expuestos por Buses de Servicios (OSB).
%
El Cat\'alogo recupera la lista de Buses de Servicios (OSBs) que conoce
(\veaseUC{136}) y
solicita a cada Bus la lista de Proxy Services que expone.
%
Luego agrega un registro por servicio (URI/Bus) que a\'un no lo tiene,
marca aquellos que ya no aparecen en un Bus (no son eliminados) y
finalmente asocia la fecha de \'ultima actualizaci\'on a cada servicio registrado.


%     .  .      .  .      .  .      .  .      .  .      .  .      .  .      .  .
\subsubsection*{Resultado deseado}
\paragraph{}
Los Proxy Services expuestos por los OSBs quedan repertoriados en el Cat\'alogo.


%     .  .      .  .      .  .      .  .      .  .      .  .      .  .      .  .
\subsubsection*{Objetivos del usuario}
\begin{itemize}
    \item Identificar los servicios expuestos por un OSB (los Proxy services)
    \item Apoyar la gesti\'on del ciclo de vida de los servicios.
\end{itemize}


%     .  .      .  .      .  .      .  .      .  .      .  .      .  .      .  .
\subsubsection*{Dependencias}
\begin{tabular}{lp{.8\textwidth}}
    Complementa: & \ucCXxxiFullName{} (sec. \ref{uc-131}). \\
\end{tabular}


%     .  .      .  .      .  .      .  .      .  .      .  .      .  .      .  .
%\subsubsection*{Pre-condiciones}


%     .  .      .  .      .  .      .  .      .  .      .  .      .  .      .  .
\subsubsection*{Escenarios}
\paragraph{}
Para evitar confundir servicios con una misma URI\footnote{
    T\'ecnicamente el Proxy Endpoint URI (ej. \url{/CRM/ProductCatalog/...})
    que contiene solamente el URL path,
    sin el esquema, host y puerto (i.e. sin \url{http://OSBMovilProd07})
    Sino los servicios balanceados aparecer\'ian duplicados.
}
es necesario identificar cada servicio por medio del Bus de Servicios que lo expone
y de su URI. 
En lo que sigue la palabra ``servicio'' se refiera a ``URI/Bus''.

\begin{itemize}
    \item
        Cuando un servicio ya se encuentra registrado en el Cat\'alogo y 
        el bus que lo expone es el mismo, no se registra cambio.
    \item 
        Cuando un servicio expuesto por un bus de servicios no est\'a en el Cat\'alogo,
        entonces es registrado.
    \item 
        Cuando una URI ya se encuentra en el Cat\'alogo pero asociada a otro Bus,
        se registra una nueva entrada URI/Bus.
    \item 
        Cuando una URI registrada en el Cat\'alogo
        ya no es expuesta por un Bus de Servicios
        que le est\'a asociado en Cat\'alogo,
        se marca el servicio URI/Bus como ya no expuesto por el Bus correspondiente.
    \item 
        Cuando la no es posible obtener las listas de servicios de alg\'un Bus de Servicios,
        el Cat\'alogo ofrece una visi\'on parcial. P.ej. servicios pueden haber sido agregados
        o dados de baja en el Bus cuya informaci\'on no est\'a disponible.\\
        Para evitar confusiones, el Cat\'alogo registra la fecha y hora
        del \'ultimo cambio en el registro del servicio.
        \PENDIENTE{Validar: esa fecha es distinta para cada grupo modificable en el Cat\'alogo}.
    \item 
        Cuando un servicio es migrado de un Bus de Servicio,
        la informaci\'on registrada en los escenarios anteriores permite al menos identificar
        su baja en un Bus de Servicios y su alta en el otro.\\
        %
        Es responsabilidad del Arquitecto colocar otras marcas si desea identificar
        expl\'icitamente la migraci\'on (p.ej. v\'ia uc-21, sec.\ref{uc-21}).
    \item 
        De manera similar a una migraci\'on, 
        cuando un servicio es renombrado, 
        la informaci\'on registrada en los escenarios anteriores permite al menos identificar
        la baja de antigua URI, en un Bus de Servicios que lo expone, 
        y el alta de su nuevo nombre.\\
        %
        Es responsabilidad del Arquitecto colocar otras marcas si desea identificar
        expl\'icitamente la un cambio de nombre (p.ej. v\'ia uc-21, sec.\ref{uc-21}).
    \item 
        Cuando se interrumpe la petici\'on o la informaci\'on llega incompleta,
        el Cat\'alogo no es modificado.\\
        \PENDIENTE{`?Dejar alguna marca, notificar algui\'en?}
\end{itemize}


%     .  .      .  .      .  .      .  .      .  .      .  .      .  .      .  .
%\subsubsection*{Aspectos t\'ecnicos}
