%-------------------------------------------------------------------------------
\section{(CMP-01) main}\label{sec:cmp-01}
%\begin{tabular}{l}
%  \textcolor{gray}{V\'ease CMP-08, sec.\ref{sec:cmp-08}.}\\
%\end{tabular}

\paragraph{}
Se ocupa el command pattern junto a inyecci\'on de dependencias
para agregar f\'acilmente distintos formatos de informaci\'on de servicios
a ser cargados en el cat\'alogo.

\paragraph{}
La figura \ref{fig:cmp01-main} muestra c\'omo se inician los batches:

\begin{itemize}
    \item \verb|CatalogLoader| es una cascara vac\'ia que solamente provee el \verb|main|
    \item \verb|CatalogLoaderOptions|
      \begin{itemize}
      \item lee y verifica los argumentos pasados por el usuario 
      \item instanciar los objetos derivados de las opciones seleccionadas
      \end{itemize}
    \item \verb|ServiceInfoKind|
      \begin{itemize}
      \item es una enumeraci\'on que identifica las clases/categor\'ias
            de informaci\'on conocidas por la base de datos del cat\'alogo
              (taxonom\'ia, OSB proxy services, access logs, \ldots)
      \item permite distinguir las tablas a llenar
      \item permite determinar la informaci\'on a proporcionar en los archivos 
            que contienen la informaci\'on a cargar
      \end{itemize}
    \item \verb|AppComposer| 
      \begin{itemize}
      \item es el punto central para configurar las dependencias en esta la aplicaci\'on
      \item llena el container que se le presenta especificando las dependencias entre objetos
      \item en particular, all\'i se define el mapeo entre \verb|ServiceInfoKind|
            e implementaci\'on concreta de \verb|Runnable|, responsable de realizar la carga
      \end{itemize}
    \item \verb|PicoContainer| 
      \begin{itemize}
      \item guarda las dependencias entre objetos
      \item resuelve las dependencias
      \item instancia objetos con sus dependencias resueltas
      \end{itemize}
    \item \verb|ComponentAdapter| 
      \begin{itemize}
      \item es una clase an\'onima definida a la medida en el composer 
      \item maneja un tipo espec\'ifico de dependencias:
            los mapeos entre \verb|ServiceInfoKind| y \verb|Runnable|
      \item dado un \verb|ServiceInfoKind| indica cu\'al es el \verb|Runnable| asociado
      \end{itemize}
    \item \verb|Runnable| 
      \begin{itemize}
      \item es la interfaz que debe implementar cada comando de carga del cat\'alogo
      \item la clase concreta se encarga de cargar la informaci\'on de servicios
      \end{itemize}
    \item \verb|main| coordina las acciones para llevar a cabo al cargar del cat\'alogo:
      \begin{itemize}
      \item instancia el container que recibir\'a las definiciones de las dependencias 
            entre objetos
      \item solicita al composer cargar las dependencias en el container
      \item inicia el parseo de los argumentos de usuario
      \item recupera los objetos correspondientes a las opciones seleccionadas
      \item agrega esos objetos en el container de dependencias
      \item solicita al container que resuelva el \verb|Runnable| correspondiente 
            al tipo de informaci\'on (\verb|ServiceInfoKind|) especificado por el usuario
      \item ejecuta el \verb|Runnable|, carg\'andose los datos a la BD
      \end{itemize}
\end{itemize}


\begin{figure}[hbt]
  \centering
  \includegraphics[width=.8\textwidth]{cmp-01-main-vopc.png}
  \caption{(CMP-01) Main, opciones, carga de dependencias.}
  \label{fig:cmp01-main}
\end{figure}
