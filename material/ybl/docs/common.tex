% NOTA perfstats vs catalogo
% casos de uso [01,20[, [70,80[  : perfstats
%              [20,30[  [60,70[  : catalogo

% Automatic label names for figure, lists, etc.
\renewcommand{\contentsname}{\'Indice del contenido}
\renewcommand{\listfigurename}{Lista de figuras}
\renewcommand{\listtablename}{Lista de tablas}
%\renewcommand{\refname}{Referencias}
\renewcommand{\figurename}{Figura}

% Project name, version, pendings, ...
\newcommand{\prjName}{Fake DataWareHouse- Cat\'alogo}
\newcommand{\prjVersion}{v1.0-alpha}
\newcommand{\PENDIENTE}[1]{\tikz[baseline=(X.base)] \node [rounded corners, fill=blue!40] (X) {PENDIENTE}; \textcolor{blue}{#1}}
\newcommand{\DUDA}[1]{}%{\tikz[baseline=(X.base)] \node [rounded corners, fill=orange!40] (X) {DUDA}; \textcolor{red}{#1}}
\newcommand{\veaseUC}[1]{v\'ease UC-#1, sec. \ref{uc-#1}}
\newcommand{\ucRef}[1]{UC-#1, sec. \ref{uc-#1}}
% UC-131
\newcommand{\ucCXxxiNr}{UC-131}
\newcommand{\ucCXxxiActor}{Temporizador}
\newcommand{\ucCXxxiTitle}{Actualizando repertorio de servicios a partir de logs de accesos}
\newcommand{\ucCXxxiName}{\ucCXxxiActor{} \ucCXxxiTitle}
\newcommand{\ucCXxxiFullName}{\ucCXxxiNr{} \ucCXxxiName{}}
% UC-132
\newcommand{\ucCXxxiiNr}{UC-132}
\newcommand{\ucCXxxiiActor}{Temporizador}
\newcommand{\ucCXxxiiTitle}{Actualizando repertorio servicios a partir de proxy services expuestos en OSB}
\newcommand{\ucCXxxiiName}{\ucCXxxiiActor{} \ucCXxxiiTitle}
\newcommand{\ucCXxxiiFullName}{\ucCXxxiiNr{} \ucCXxxiiName{}}
