%-------------------------------------------------------------------------------
\section{Catalogando m\'ultiples fuentes de info}
\paragraph{}
Esta secci\'on presenta la expansi\'on del modelo de datos
para consolidar distingos aspectos de un mismo servicio desde diversos
fuentes de informaci\'on
(uso real desde logs de acceso,
 gesti\'on del ciclo de vida desde la planilla de taxonomi\'ia,
 declaraci\'on de proxy service desde de los buses de servicios);
para discriminar el uso real de los servicios (registrado por los logs de acceso)
vs su declaraci\'on de existencia (registrados en la taxonom\'ia y en los OSB);
para distinguir en qu\'e bus de servicio est\'an registrados, si son proxy services;
para discriminar los grupos de instancias (cluster o dominio) que albergan los servicios.


%. . . . . . . . . . . . . . . . . . . . . . . . . . . . . . . . . . . . . . . .
\subsection{Modelo conceptual}\label{sec:multi-svclist-ER}
\paragraph{}
La figura \ref{fig:db-multi-er} presenta los principales conceptos involucrados
en el manejo de informaci\'on \emph{acerca} de los servicios.

\begin{figure}[htb]
    \centering
    \includegraphics[width=.4\textwidth]{db-multi-svclist-conceptual.png}
    \caption{ER para soportar m\'ultiples fuentes de informaci\'on acerca de distintos aspectos de los servicios.}
    \label{fig:db-multi-er}
\end{figure}


%     .  .      .  .      .  .      .  .      .  .      .  .      .  .      .  .
\subsubsection{Individualizando servicios}
\paragraph{}
El URL path permite identificar servicios independientemente de
la direcci\'on IP u hostname de la instancia de servidor que lo expone
o del load-balancer detr\'as del cual se encuentra esa instancia.

\paragraph{}
Lo anterior permite reducir la complejidad de la gesti\'on de los servicios,
considerando que hay cerca de 203 instancias en producci\'on (Enero 2018).
y que, en general, un mismo servicio es desplegado en varias instancias.
%
En general un servicio es desplegado en todas las instancias de un mismo 
cluster o dominio.
Adem\'as la migraci\'on de servicios se realiza por clusters enteros
(p.ej. de la plataforma 10.2 hacia 10.3.6).
%
Distintos clusters, a su vez realizan una partici\'on de los servicios.
Ya sea para su administraci\'on, una asignaci\'on adecuada de los recursos
o incluso por seguridad.

\paragraph{}
Por ese motivo, la identificaci\'on de servicios tambi\'en considera
el \emph{grupo de nodos},
cualquiera sea el mecanismo (cluster, dominio, load-balancer, etc.)
como parte de la identificaci\'on de un servicio para su gesti\'on
a trav\'es del Cat\'alogo.


%     .  .      .  .      .  .      .  .      .  .      .  .      .  .      .  .
\subsubsection{Tipos de servicio}
\paragraph{}
Actualmente (inicio 2018),
los tipos de servicios considerados en el cat\'alogo son los servicios accesados
por HTTP, sobre las plataformas WebLogic.

\paragraph{}
Aunque hay varios tipos de servicios m\'as --MQ, bases de datos, LDAP, EJBs, etc.--
el cat\'alogo se centra inicialmente en servicios y aplicaciones web.


%     .  .      .  .      .  .      .  .      .  .      .  .      .  .      .  .
\subsubsection{Datos detallados de servicios}
\paragraph{}
La informaci\'on detallada provista acerca de los servicios depende del tipo 
de informaci\'on recabada, y de las preguntas a las cuales responde:
taxonom\'ia en excel, registros de servicios en OSBs, logs de accesos, etc.
%
Adem\'as de proporcionar informaci\'on \emph{acerca de} distintos aspectos
de los servicios considerados, y de presentar distintas estructuras de datos,
esas fuentes de informaci\'on est\'an distribu\'idas y son accesibles v\'ia distintos
protocolos.

\paragraph{}
En consecuencia, de ser necesario, se mantiene una tabla espec\'ifica por
tipo de informaci\'on a recabar acerca de los servicios.
Tambi\'en se identifica el ``tipo'' de informaci\'on y
su fuente, la que generalmente es identificable por un URL path de un cluster.

\paragraph{}
Cabe notar que no es siempre necesario crear una tabla nueva para cada tipo de informaci\'on.
P.ej. en el caso del uso de los servicios, registrado en los logs de acceso,
y en el caso de los registros de proxy services en OSBs,
la pregunta actual es saber si hay alg\'un registro de los servicios en dichos repositorios.
%
Para ello es suficiente disponer de una lista de servicios y con la fecha en la cu\'al
fueron encontrados en el repositorio correspondiente.


%     .  .      .  .      .  .      .  .      .  .      .  .      .  .      .  .
\subsubsection{Informaci\'on recabada}
\paragraph{}
El Cat\'alogo es alimentado a partir de informaci\'on recabada desde distintas fuentes
(taxonom\'ia, OSBs, logs de acceso).
Algunas de estas son m\'as confiables, otras est\'an m\'as actualizadas,
algunas presentan elementos de planes futuros y expectativas acerca de los servicios.
%
Sobre todo al recabar informaci\'on acerca de los servicios, se obtienen ``pedazos''
de informaci\'on que son consolidados en el cat\'alogo
para describir y gestionar los servicios.

\paragraph{}
Tales listados son identificados y discriminados
por medio del tipo de informaci\'on recabada,
su fuente cuando hay varias del mismo tipo (no todas contienen los mismos servicios),
la fecha y un indicador de la completitud de la informaci\'on recabada.
%
Discriminando el origen de los listados, es posible cotejar la informaci\'on recabada
corregirla, mejorando as\'i la consistencia de la informaci\'on acerca de los servicios.

\paragraph{}
Por \emph{completitud} se entiende que la lista contiene
todos los servicios registrados en la fuente de informaci\'on.
As\'i se puede deducir que un servicio que no se encuentra en el listado puede
considerarse como ya no registrado en la fuente de informaci\'on considerada.
Los OSBs y logs de acceso son completos en ese sentido,
aunque pueden cubrir distintos grupos de servicios.
Mientras que la taxonom\'ia es incompleta.
P.ej. una lectura es
``si el URL path X no se encuentra en osbentelprod, es que no proporcionado por ese OSB'',
``si el URL path Y no se encuentra en los logs de acceso (Elasticsearch), es que no ha sido usado''.

\paragraph{}
Las principales fuentes de informaci\'on, explotables actualmente,
acerca de los servicios son las siguientes:

\paragraph{}
\begin{tabular}{lp{.7\textwidth}}
    Logs de acceso
    &
    Los logs de acceso dan cuenta del \emph{uso real} de los servicios.
    T\'ecnicamente son cargados por Logstash en Elasticsearch donde pueden
    realizarse consultas.
    \\
    Taxonom\'ia
    &
    Arquitectura mantiene una taxonom\'ia de servicios web,
    principalmente proxy services registrados en OSB.
    %
    La taxonom\'ia es guardada en una planilla Excel y es cargada al cat\'alogo
    (release-1).
    \\
    Registro OSB
    &
    Entel tiene varios buses de servicio OSB y ALSB, los cuales exponene Proxy Services,
    escondiendo su implementac\'on (Business Services), as\'i como composici\'on y 
    transformaciones. Los Business Services est\'an constituidos por servicios web, 
    MQ, \ldots
\end{tabular}


%     .  .      .  .      .  .      .  .      .  .      .  .      .  .      .  .
\subsubsection{Grupos actuales}
\paragraph{}
Lo que sigue entrega cifras que fundamentan el uso del dominio como dato necesario
para discriminar la informaci\'on acerca de los servicios,
de manera de poder gestionarlos adecuadamente.

\paragraph{}
Entel aplica un esquema estandarizado de nombramiento de instancias de servidores.
%
T\'ecnicamente al extraer los access logs para alimentar Elasticsearch, 
el Logstash de Entel reconstruye el dominio al cual pertenece cada URL invocada.

\paragraph{}
Se asume que es dato corresponde a los dominios WebLogic y/o clusters de servidores.

\paragraph{}
La tabla a continuaci\'on indica la cantidad m\'axima de dominios en las cuales
distintas URL pueden aparecer. P.ej. \url{/testApp} est\'a instalada en 32 dominios.
%
Se consideran los 5.683 URL paths \emph{agrupados} (\verb|url_path_virtual|)
que aparecen en los access logs
de todos los OSB, ALSB y servidores WebLogic en producci\'on (Enero 2018).

\paragraph{}
\begin{tabular}{rlrl}
    3.386 & url paths se encuentran en & 1 & dominio\\ 
    2.287 & url paths se encuentran en & 2 & dominios\\ 
        9 & url paths se encuentran en & 3 & dominios\\ 
        1 & url path se encuentra en   & 32 & dominios\\
\end{tabular}

\paragraph{}
Los url paths que est\'an en 2 dominios se encuentran en el dominio principal y
en el dominio del Site B correspondiente, lo que parece indicar un proceso de migraci\'on.
%
Este es precisamente el tipo de informaci\'on relevante para la gesti\'on del ciclo
de vida de los servicios.


%. . . . . . . . . . . . . . . . . . . . . . . . . . . . . . . . . . . . . . . .
\subsection{Modelo f\'isico}\label{sec:multi-svclist-phys}
\paragraph{}
Esta secci\'on describe los principales elementos del modelo f\'isico.
La figura \ref{fig:db-multi-load} presenta los stored procedures y tablas
que dan soporte al modelo conceptual
presentado (sec. \ref{sec:multi-svclist-ER}).
La figura \ref{fig:db-multi-tables} identifica las vistas y tablas necesarias para
desplegar el Cat\'alogo al usuario.


\begin{figure}[htb]
    \centering
    \includegraphics[width=\textwidth]{db-multi-svclist-load.png}
    \caption{Programas y stored procedures que alimentan las tablas desde m\'ultiples fuentes de informaci\'on, para presentar distintos aspectos de los servicios.}
    \label{fig:db-multi-load}
\end{figure}


\begin{figure}[htb]
    \centering
    \includegraphics[width=\textwidth]{db-multi-svclist-tables.png}
    \caption{Vistas y tablas que participan del despliegue de la informaci\'on de los servicios en el Cat\'alogo}
    \label{fig:db-multi-tables}
\end{figure}


%     .  .      .  .      .  .      .  .      .  .      .  .      .  .      .  .
\subsubsection{Listados hist\'oricos}
\paragraph{}
La informaci\'on acerca de los servicios es \emph{hist\'orica}.
Ello queda reflejado en la columna \verb|FECH_LISTA| de la tabla
\verb|GMD_NEG_LSTINFO| y por el hecho que esta tabla es parte de la llave primaria
de las tablas de detalle (p.ej. \verb|GMD_NEG_TAXONOMIA_SERVICIO|).


%     .  .      .  .      .  .      .  .      .  .      .  .      .  .      .  .
\subsubsection{Detalle por fuentes de informaci\'on}
\paragraph{}
Cada fuente de informaci\'on proporciona ciertos detalles acerca de los servicios.
Los elementos comunes son capturados en la tabla \verb|GMD_NEG_SVCDETALLE|,
mientras que se debe crear una tabla espec\'ifica para los detalles espec\'ificos.

\paragraph{}
Tal es el caso para la informaci\'on de servicios proveniente de
la taxonom\'ia (tabla \verb|GMD_NEG_SVCTAXONOMIA|)
y de los OSB (tabla \verb|GMD_NEG_SVCOSB|).
Sin embargo (por ahora) no es necesario para la informaci\'on de servicios
extra\'ida de los logs de accesos, en ese caso tabla com\'un es suficiente.

\paragraph{}
Tambi\'en cabe notar que la misma tabla \verb|GMD_NEG_URLPATH| permite identificar los servicios
y las fuentes de informaci\'on acerca de los mismos sin conflictos (ya que las fuentes de informaci\'on tambi\'en son servicios).


%     .  .      .  .      .  .      .  .      .  .      .  .      .  .      .  .
\subsubsection{Identificaci\'on de fuente de informaci\'on}
\paragraph{}
Cada fuente de informaci\'on es identificada
por la clase de informaci\'on
(tabla \verb|GMD_NEG_CLASEINFO|, columna \verb|VLOR_LSTCLASEINFO|)
y servidor que guarda esa informaci\'on,
el cual pertenece a dominio (cluster) y tiene un URL path de administraci\'on
(ocup\'andose la tabla \verb|GMD_NEG_URLPATH| para identificarlo).
%
La tabla \ref{tab:svc-list-src} muestra c\'omo reconocer algunas fuentes actuales.

\paragraph{}
\begin{table}[htb]
    \centering
    \caption{Individualizando fuente de info acerca de los servicios ocupando tablas asociadas a GMD\_NEG\_LISTADO\_SERVICIOS} 
    \begin{tabular}{llll}
       \toprule 
        \verb| VLOR_LSTCLASEINFO| & \verb| NOMB_GRUPONODOS| & \verb| NOMB_URLPATH| \\
        \midrule
        Taxonomia Arquitectura & local file           & \verb|TAXONOMIA_CONSOLIDADA.xlsx| \\
        Access Logs              & \verb|elasticsearch| & \verb|/logstash-*/_search|\\
        OSB Proxy Services       & osbmovilprod         & \verb|/sbconsole| \\
        OSB Proxy Services       & alsbsrvprod          & \verb|/sbconsole| \\
        \ldots & & \\
       \bottomrule 
    \end{tabular}\label{tab:svc-list-src}
\end{table}


%     .  .      .  .      .  .      .  .      .  .      .  .      .  .      .  .
\subsubsection{Cargando la informaci\'on de servicios}
\paragraph{}
Para cargar la informaci\'on desde las distintas fuentes, se ocupa una tabla temporal
(\verb|GMD_TMP_IMPORTA| en fig. \ref{fig:db-multi-load}).
Dependiendo de la clase de informaci\'on, se cargan m\'as o menos campos.
Luego un stored procedure realiza los chequeos necesarios, insertando valores faltantes
en tablas de valores comunes para finalmente cargar la informaci\'on m\'as espec\'ifica.
