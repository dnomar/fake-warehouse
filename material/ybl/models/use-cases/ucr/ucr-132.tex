%-------------------------------------------------------------------------------
\section{UCR-132: \ucCXxxiiTitle{}(realizaci\'on)}\label{sec:ucr-132}
\textcolor{gray}{V\'ease la especificaci\'on del caso de uso, sec. \ref{uc-132}}


%. . . . . . . . . . . . . . . . . . . . . . . . . . . . . . . . . . . . . . . .
\subsection{UCR-132 VOPC}
\paragraph{}
La figura \ref{fig:ucr-132-vopc} muestra la composici\'on del batch de actualizaci\'on
de la lista de URIs. Algunas consideraciones:

\begin{itemize}
    \item
        El Cat\'alogo mantiene la lista de URL de conexi\'on (t3) a cada cluster OSB
        desde los cuales s puede actualizar autom\'aticamente la lista de Proxy services
        registrados en el cat\'alogo.
    \item
        El nodo de administraci\'on de cada cluster OSB tiene el registro
        de los Proxy services.
    \item
        La extracci\'on de las URI de los Proxy services guarda:
        la URL del cluster OSB que lo expone, el nombre y la URI de cada Proxy service.
        As\'i es posible distinguir servicios duplicados (en distintos OSBs).
    \item
        La carga de los Proxy services a actualizar en el Cat\'alogo se realiza por medio de
        por una tabla paso y un stored procedure.
    \item
        La figura no muestra las tablas actualizadas por \verb|sp_update_osb_proxy_svc|.
    \item
        Las credenciales para el acceso a los servidores se mantienen en una archivo
        CSV encriptado.
        El proceso que gatilla el batch es el due\~no de la clave para desencriptar
        el archivo de credenciales.
        \\
        La URL del servidor sirve de llave:
        cada fila tiene la URL del servidor, el usuario y la clave.
\end{itemize}

\begin{figure}[hbtp]
    \centering
    \includegraphics[width=\textwidth]{ucr-132-vopc.png}
    \caption{UC-132 VOPC.}\label{fig:ucr-132-vopc}
\end{figure}

