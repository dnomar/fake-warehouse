\documentclass[11pt,letterpaper]{report}

% NEXT
% ====
%
%
%
%


%\usepackage[paper=letterpaper,left=3cm,top=2cm,bottom=2cm,twoside=false]{geometry} %usar showframe para ver pos.
\usepackage[paper=letterpaper,top=2.5cm,left=3cm,right=3cm,bottom=3cm,twoside=false]{geometry} %usar showframe para ver pos.
\usepackage{fancyhdr}
\pagestyle{fancy}
\usepackage{appendix}
\usepackage{graphicx}
\usepackage{minted}
\usepackage{xcolor}
\usepackage{tikz}
%
% \usepackage{amsmath}
% \usepackage{amssymb}


% Prevents floating graphics or tables from floating into the next section.
\usepackage[section]{placeins}
\usepackage{import}
\usepackage{hyperref}
\hypersetup{
    pdfborder={0 0 0}
}
\usepackage{lineno}
\usepackage{booktabs}
\usepackage{listings}
\usepackage{titling}
\usepackage{verbatim}

\import{.}{common.tex}

\fancyhead{}
%\fancyhead[RE,RO]{\includegraphics[scale=.7]{my-logo-161x82.png}}
\fancyhead[LE,LO]{\slshape \leftmark}

% ======================================================================
\begin{document}
% ======================================================================

\begin{titlepage}
    \thispagestyle{empty}
    \pagenumbering{gobble}
    \title{\Huge\prjName{}\\[15px]\huge Documento de dise\~no\\{\tiny \prjVersion{}}\\[40px]}
    \author{Y. Bossel - Everis}
%    \date{\today}
\end{titlepage}
\maketitle

\newpage
\pagenumbering{roman}

\import{./}{design-changelog.tex}

\newpage
\tableofcontents
\listoffigures
%\listoftables

\newpage
\pagenumbering{arabic}

%===============================================================================
\chapter{Acerca del proyecto}
% Project Definition (as a SOW preparation)
\paragraph{}
El proyecto tiene como objetivo \ldots


%%------------------------------------------------------------------------------
%\section{Foco de la soluci\'on}
%% Foco en la soluci\'on
%%  (a) los elementos que la influencian, impactan, le dan forma
%%  (b) forma que tiene y descripcion de su forma desde varias perspectivas
%%  (c) validacion rnf, funcional, riesgos
%%
%% Otros documentos tienen los requerimientos, modelo del dominio, etc.
%
%%..............................................................................
%\subsection{Prop\'osito}
%
%%..............................................................................
%\subsection{Alcance}
%
%%..............................................................................
%\subsection{Referencias}
%
%%..............................................................................
%\subsection{Resumen del documento}
%
%
%%------------------------------------------------------------------------------
%\section{Antecedentes}
%%			← stay focused
%
%%..............................................................................
%\subsection[Contexto del proyecto]{Resumen del contexto del project}
%
%%..............................................................................
%\subsection{Comienzo del project}
%
%%..............................................................................
%\subsection[Planteamiento del problema]{Resumen del planteamiento del problema}
%
%%..............................................................................
%\subsection{Caracter\'isticas de la soluci\'on}
%
%%..............................................................................
%\subsection{Valor de negocio de la soluci\'on}
%
%
%%..............................................................................
%\subsection{Dimensiones y desempe\~no actuales}
%%		← ref to size & perf req + Support of ~


%===============================================================================
\chapter{Modelo de casos de uso}\label{chap:uc-model}

\import{../models/use-cases/specs/}{use-cases-subsystems_desc.tex}

\paragraph{}
A continuaci\'on se describe el m\'odulo de Cat\'alogo de Servicios.
Sus casos de uso se encuentran graficados en la figura
\ref{fig:use-cases-catalog}.

\begin{figure}[htbp]
    \centering
    \includegraphics[width=.6\textwidth]{../models/use-cases/specs/use-cases-subsystems.png}
    \caption{Grupos de funcionalidades.}\label{fig:use-cases-subsystems}
\end{figure}

\begin{figure}[htbp]
    \centering
    \includegraphics[width=\textwidth]{../models/use-cases/specs/use-cases-catalog.png}
    \caption{Casos de uso -- cat\'alogo.}\label{fig:use-cases-catalog}
\end{figure}


%-------------------------------------------------------------------------------
\section{Casos de uso}
\paragraph{}
Esta secci\'on describe cada caso de uso. 
M\'as que una especificaci\'on normativa,
tiene por objetivo decribir el funcionamiento esperado del Cat\'alogo,
identificando los objetivos de los actores, el resultado esperado y
los principales escenarios de uso.

\import{../models/use-cases/specs/}{uc-131.tex}
\import{../models/use-cases/specs/}{uc-132.tex}


%===============================================================================
\chapter{Vista de componentes}
\paragraph{}
A continuaci\'on son presentado succintamente algunos elementos comunes de
la aplicaci\'on web del Cat\'alogo de servicios.


%-------------------------------------------------------------------------------
\section{Packages}

\begin{figure}[htb]
    \centering
    \includegraphics[width=\textwidth]{../models/component-model/deps-packages.png}
    \caption{Capa web (release 1), dependencias entre packages.}
    \label{fig:pkg-deps}
\end{figure}

\import{../models/component-model/}{deps-packages_desc.tex}


%-------------------------------------------------------------------------------
\import{../models/component-model/}{cmp-01-main-and-di.tex}
\import{../models/component-model/}{cmp-02-batches.tex}


%===============================================================================
\chapter{Vista de datos}
\paragraph{}
A continuaci\'on son presentado los elementos del modelo de datos.

\import{../models/data-model/}{db-multi-svclist.tex}



%===============================================================================
\chapter{Realizaci\'on de casos de uso}\label{chap:ucr}

\import{../models/use-cases/ucr/}{ucr-131.tex}
\import{../models/use-cases/ucr/}{ucr-132.tex}


%= - = - = - = - = - = - = - = - = - = - = - = - = - = - = - = - = - = - = - = -
\appendix

%===============================================================================
\chapter{Configuraci\'on}
\paragraph{}
A continuaci\'on se decriben los principales elementos a configurar.
\ldots


%===============================================================================
\chapter[Batches setup]{Setup de batches de extracci\'on y carga de servicios}
\textcolor{gray}{V\'ease tambi\'en componentes sec. \ref{sec:cmp-02}}
\paragraph{}
\verbatiminput{../code/scripts/README}


% ======================================================================
\end{document}
% ======================================================================
